% \part{Особенности архитектурно-независимой парадигмы параллельного программирования}
%\chapter[Факторы архитектурной независимости...]{Факторы, определяющие архитектурную независимость параллельных программ}

\chapter[Особенности АНППП]{Особенности архитектурно-независимой парадигмы параллельного программирования}

Поддержка архитектурной независимости на уровне моделей вычислений и языков программирования, связана с подходами, обеспечивающими специфические методы хранения данных, формируемых в ходе выполнения программы, а также независимостью стратегии управления вычислениями~\cite{strat} от реальных вычислительных ресурсов.

Независимость хранения данных от памяти поддерживается функциональной парадигмой программирования, ориентированной на представление программ в виде взаимодействующих функций. В отличие от императивного подхода память данных представлена в неявном виде. Использование рекурсий вместо итераций позволяет избавиться на уровне описания алгоритмов от повторного использования переменных. Эти решения во многом обеспечивает архитектурную независимость программ и реализованы в разнообразных языках функционального программирования (ЯФП). Однако большинство таких языков имеют ограничения по неявному представлению параллелизма задачи, что обуславливается особенностями представления структур данных в виде списков с последовательным доступом к их элементам. Наличие в списке только варианта доступа к голове и хвосту не позволяет организовать параллельные вычисления непосредственно на основе текущих реализаций. Поэтому для поддержки параллелизма в ЯФП обычно используется явное управления, на основе которого создаются потоки или процессы, что ведет к произвольному воздействию на параллелизм со стороны разработчика и является фактором, определяющим наличие архитектурной зависимости.

Стратегии управления вычислениями по готовности данных (dataflow control) позволяют неявно описывать параллелизм. Одной из первых таких моделей вычислений (МВ) является модель Денниса~\cite{dennis}. Она легла в основу ряда специализированных процессоров с различной архитектурой. Можно отметить различные языковые средства, использующие управление по готовности данных, и применяемых для программирования различных архитектур. Например: Sisal~\cite{sisal}, Colamo~\cite{colamo}, LuNA~\cite{luna}. В ряде систем программирования управление потоками данных сочетается с функциональным стилем. Однако для многих МВ проблематично говорить об архитектурной независимости, что зачастую связано с ориентацией языков программирования и методов их трансформации на определенные архитектурные решения.  В большинстве из них до конца не проработана концепция неограниченного параллелизма. Также часто управление по готовности данных сочетается с использованием явного управления или с необходимостью управления ограниченными ресурсами.

Наряду с функциональным подходом и управлением по готовности данных ряд задач, связанных с архитектурно-независимым представлением параллелизма, можно решить, используя специальные структуры, которые не только содержат данные, но и обеспечивают поддержку их разнообразного параллельного поведения. При этом отличия в поведении вводимых конструкций определяет подходы к различной организации параллелизма. Инкапсуляция динамического поведения данных внутри специальных структур (динамически формируемых данных) позволяет убрать из программ явное управление вычислениями, обычно применяемое в императивных языках программирования, заменяя его на взаимодействие с функциями и другими структурами неявным управлением по готовности данных. Применение параллельной рекурсии позволяет рассматривать программу как описание активностей, выполняемых в неограниченных вычислительных ресурсах, формируемых неявно по мере надобности. Подобный подход на уровне языка программирования позволяет использовать его в качестве архитектурно--независимого. Вместе с тем это предъявляет особые требования к трансформации программ с таких языков в архитектурно--зависимые программы.

Описанный подход был предложен при разработке языка функционально--потокового параллельного программирования Пифагор, разработанного на основе функционально--потоковой модели параллельных вычислений (ФПМПВ)~\cite{legalov2005,asynch}. Соответствующие структуры представлены в нем как списки данных, параллельные списки, задержанные списки, асинхронные списки. Каждый вид списков задает свои методы группировки данных и способы управления по готовности этих данных. К недостаткам предложенных конструкций, как и языка в целом можно отнести динамическую типизацию атомарных типов, а также динамическое формирование списков в ходе вычислений, что не позволяет во время компиляции программы сформировать эффективное выходное представление.

Необходимость использования статической типизации для повышения эффективности трансформации в реальные архитектуры была подтверждена в ходе реализации ряда проектов по трансформации функционально-потоковых параллельных программ:
\begin{itemize}
	\item при преобразовании в топологию ПЛИС~\cite{romanova2022};
	\item при трансформации в статически типизированный императивный язык программирования~\cite{transform}.
\end{itemize}
Для получения требуемых решений в промежуточное представление, порождаемое компилятором языка Пифагор, пришлось вводить дополнительные описания, определяющие типы обрабатываемых данных.

Таким образом, для решения задачи эффективной трансформации архитектурно--не\-за\-ви\-си\-мой параллельной программы в программу для реальной целевой архитектуры необходимо совместно использовать подходы, которые по отдельности не решают целевую задачу: \begin{enumerate}
	\item неявное управление вычислениями по готовности данных (dataflow control);
	\item функциональную парадигму программирования;
	\item специальные структуры данных, ориентированные на представление различных видов параллелизма;
	\item статическую типизацию данных.
\end{enumerate}
Их совместное использование отражается как на модели параллельных вычислений, так и инструментальных средствах, создаваемых на ее основе. Ориентация на архитектурную независимость по сути ведет к формированию предметно-ориентированной модели параллельных вычислений и создаваемому на ее основе предметно ориентированному языку архитектурно-независимого параллельного программирования со специфическим для него набором артефактов и их семантикой.