\chapter*{Введение}
\label{intro}
\addcontentsline{toc}{chapter}{Введение}

Разработка параллельных программ в настоящее время ориентирована на использование методов, учитывающих особенности архитектур целевых вычислительных систем. Это связано со стремлением повысить производительность параллельных вычислений. Исследования в области архитектурно--независимого параллельного программирования еще не сформировались до окончательных практических решений, продолжаясь по следующим направлениям:
\begin{itemize}
	\item автоматическое или полуавтоматическое распараллеливание последовательных программ с последующей их трансформацией под целевую архитектуру~\cite{seq-par-01};
	\item разработка программ или алгоритмов, обладающий неограниченным параллелизмом, определяемым решаемой задачей, с последующим <<сжатием>> этого параллелизма в соответствии с ограничениями, определяемыми целевой архитектурой~\cite{aipp}.
\end{itemize}

При любом из этих подходов между написанной программой и реальной параллельной вычислительной системой (ПВС) существует семантический разрыв. При трансформации в машинный код происходит потеря эффективности и сбалансированности, так как характеристики <<архитектурно--независимого>> последовательного или параллельного алгоритма вступают в противоречие с организации вычислений в конкретных ПВС. Именно поэтому неограниченный параллелизм зачастую вручную <<сжимается>> при доводке программы под особенности вычислителя, определяя тем самым подход, противоположный распараллеливанию последовательных программ. Ручная трансформация ведет к потере эффективности процесса разработки параллельного программного обеспечения, не позволяя писать программу один раз и для различных архитектур. В связи с этим актуальной является задача поиска моделей параллельных вычислений и построение на их основе языковых и инструментальных средств, обеспечивающих эффективную подстройку параллелизма однажды написанной программы под различные вычислительные ресурсы.

Одним из путей, определяющим возможность более эффективной трансформации к реальным архитектурам, является использования в языках программирования статической типизации данных, которая обеспечивает эффективную компиляцию в бестиповое представление на уровне системы команд~\cite{types}. Использование этого подхода широко распространено как в последовательном, так и в параллельном программировании. Однако, для написания архитектурно-независимых параллельных программ только статической типизации недостаточно. Необходимо также учитывать особенности конструкций, описывающих параллелизм, динамические характеристики которых могут затруднять трансформацию в машинный код существующих ПВС. 


\section*{От автора}
\label{intro:author}

Концепция архитектурно-независимого параллельного программирования (АНПП) на текущий момент не находит приверженцев. Мои прогнозы относительно путей развития методов параллельного программирования, сформулированные в конце девяностых, не оправдались. По прошествии практически уже почти тридцати лет разработка параллельных программ остается привязанной к особенностям конкретных параллельных вычислительных систем (ПВС). Меняются архитектуры как последовательных, так и параллельных компьютеров, но все также неизменным остается подход, направленный на написание программ, ориентированных для каждой из таких систем и не рассматривающий общую теорию параллельных программ в качестве первоосновы.

В настоящее время очень мало работ в области анализа особенностей построения параллельных программ как некоторых универсальных конструкций, напрямую не связанных с реальными вычислителями. Среди них следует отметить работы Алеевой~В.Н.~\cite{aleeva-pct-2019}, представляющих параллельные вычисления на основе Q-детерминанта. Также близкую тему затрагивают работы по ресурсно-независимому параллельному программированию, ориентированному в то же время на спецвычислители, разработанные с применением ПЛИС~\cite{levin-pct-2018}. Но больше ничего конкретного назвать не могу. Поэтому остается верить, что то, чем в рамках научных исследований я занимаюсь, не является тупиковой ветвью и когда-нибудь (желательно как можно быстрее, пока я нахожусь в ясном уме и трезвой памяти) выстрелит.

\debate[AL]{Скорее всего в дальнейшем этот обзор расширится но не сильно}

Опираясь на эту веру, я продолжаю развивать концепцию архитектурно-независимого параллельного программирования на основе функционально-потоковой парадигмы, основной идеей которой является написание программ совершенно не связанных с ресурсными ограничениями реальных ПВС. В рамках этой концепции ключевым аспектом является эффективная трансформация архитектурно-независимых параллельных программ в уже существующие архитектуры, а не их непосредственное выполнение на специально создаваемых системах. На текущий момент разработан язык функционально-потокового параллельного программирования (ЯФППП) Пифагор~\cite{legalov-vt-2005}. Однако в его основе лежит динамическая система типов, что не обеспечивает гибкой трансформации в реальные параллельные архитектуры. Как и для любого динамически типизированного языка в данном случае имеются много накладных расходов, ведущий к резкому уменьшению производительности вычислений. Помимо этого операторы функционально-потоковой модели параллельных вычислений (ФПМПВ) в большинстве своем ориентированы на динамическую организацию данных и хранение элементов произвольного типа, что также является ограничивающим фактором для преобразований архитектурно-независимых параллельных программ в ресурсно-зависимую форму, характерную для современных вычилительных систем (ВС). Поэтому для дальнейшего продолжения работ необходимо сформировать концепцию модели вычислений, которая бы обеспечила поддержку более эффективных трансформаций в архитектуры реальных ПВС для создаваемого на основе этой модели языка программирования.

Подобные концепции в настоящее время обычно опираются на статическую типизацию и организацию данных, размерность которых и внутренняя структура в основном определяются на этапе компиляции. Внедрение этих идей требует пересмотра ряда понятий лежащих в основе уже разработанных ФПМПВ и ЯФППП. Поэтому необходима разработка другой модели, ориентированной на поддержку новых идей которая бы обеспечила основу для создания соответствующего языка. Ниже рассматриваются концепции статически типизированной модели функционально-потоковых параллельных вычислений (СТМФППВ), которую предполагается положить в основу соответствующего статически типизированного ЯФППП (СТЯФППП) Smile. Название языка выбрано исходя из давно возникших ассоциаций, что многие из формируемых в процесс написания программ сочетаний символов образуют комбинации, напоминающие смайлики. Это определило выбор смайлика <<:-)>> в качестве символа языка.

\subsection*{О чем эта книга}

Данный план описывает приблизительное содержание работы. В ходе написания он может изменяться. Пока же он представляет некоторый аналог персонального мозгового штурма.

\begin{enumerate}
	 \item  Статически типизированная модель модель функционально-потоковых параллельных вычислений.
 	\begin{enumerate}
   		\item Базовые положения.
   		\item Общие принципы организации модели.
  	\end{enumerate}
   	\item Язык программирования Smile. Предварительное описание.
	\item Семантика оператора интерпретации.
	\item Синтаксис языка программирования Smile.
\end{enumerate}

