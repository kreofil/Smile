\begin{thebibliography}{00}

\bibitem{seq-par-01}
Штейнберг, Б. Я. Преобразования программ --- фундаментальная основа создания оптимизирующих распараллеливается компиляторов / Б. Я. Штейнберг, О. Б. Штейнберг // Программные системы: теория и приложения. – 2021. – Т. 12. – № 1(48). – С. 21-113. – DOI 10.25209/2079-3316-2021-12-1-21-113. – EDN FZFEPX.

\bibitem{aipp}
Legalov A.I., Vasilyev V.S., Matkovskii I.V., Ushakova M.S. A Toolkit for the Development of Data-Driven Functional Parallel Programmes. In: Sokolinsky L., Zymbler M. (eds) Parallel Computational Technologies. PCT 2018. Communications in Computer and Information Science, vol 910. Springer, Cham, pp 16-30. DOI https://doi.org/10.1007/978-3-319-99673-8\_2.

\bibitem{types}
Pierce Benjamin C. Types and Programming Languages // The MIT Press, 2002.

\bibitem{aleeva-pct-2019}
Алеева В.Н., Алеев Р.Ж. Применение Q-детерминанта численных алгоритмов для параллельных вычислений // Параллельные вычислительные технологии – XIII международная конференция, ПаВТ’2019, г. Ростов-на-Дону, 2–4 апреля 2019 г. Короткие статьи и описания плакатов. Челябинск: Издательский центр ЮУрГУ, 2019. С. 133-145. [Текст в формате pdf: \url{http://omega.sp.susu.ru/pavt2019/short/006.pdf}

\bibitem{levin-pct-2018}
High-Performance Reconfigurable Computer Systems with Immersion Cooling (p. 62-76).

\bibitem{legalov-vt-2005}
Легалов А.И. Функциональный язык для создания архитектурно-независимых параллельных программ. – Вычислительные технологии, № 1 (10), 2005. С. 71-89.

\bibitem{strat}
Легалов\,А.И. Об управлении вычислениями в параллельных системах и языках программирования --- Научный вестник НГТУ, № 3 (18), 2004. С. 63-72.

\bibitem{dennis}
Деннис Дж.Б, Фоссин Дж.Б., Линдерман Дж.П.
Схемы потока данных. --- Теория программирования, Т2, 1972, с. 7-43.

\bibitem{sisal}
Касьянов\,В.Н., Стасенко\,А.П. Язык программирования Sisal 3.2 // Методы и инструменты конструирования программ. --- Новосибирск: ИСИ СО РАН, 2007. --- С.~56--134.

\bibitem{colamo}
Каляев\,И.А., Левин\,И.И. Реконфигурируемы вычислительные системы на основе ПЛИС. --- Ростов-на-Дону: Издательтво ЮНЦ РАН, 2022. --- 475 с.

\bibitem{luna}
Перепелкин~В.А.
Система LuNA автоматического конструирования параллельных программ численного моделирования на мультикомпьютерах --- журнал "Проблемы информатики", 2020, № 1. --- с.66-74. 
DOI:~\href{https://doi.org/10.24411/2073-0667-2020-10004}{10.24411/2073-0667-2020-10004}

\bibitem{legalov2005}
Легалов~А.\,И. Функциональный язык для создания архитектурно-независимых параллельных программ.~//
Вычислительные технологии №~1~(10). 2005. C.~71--89

\bibitem{asynch}
Легалов~А.И., Редькин~А.В., Матковский~И.В. Функционально-потоковое параллельное программирование при асинхронно поступающих данных. // Па\-рал\-лель\-ные вычислительные технологии (ПаВТ'2009): Труды международной научной конференции (Нижний Новгород, 30 марта – 3 апреля 2009 г.). 
--- Че\-ля\-бинск: Изд. ЮУрГУ, 2009. --- С. 573--578. (Электронное издание)

\bibitem{romanova2022}
Romanova~D.\,S., Nepomnyashchiy~O.\,V., Ryzhenko~I.\,N., Legalov~A.\,I., Sirotinina~N.\,Y. Parallelism reduction method in the high-level VLSI synthesis implementation. // Trudy ISP RAN/Proc. ISP RAS. 2022. Vol.~34, No.~1. P.59--72. DOI:~\href{https://doi.org/10.15514/ISPRAS-2022-34(1)-5}{10.15514/ISPRAS-2022-34(1)-5}

\bibitem{transform}
Васильев~В.\,С., Легалов~А.\,И., Зыков~С.\,В. Трансформация функционально-потоковых параллельных программ в императивные. // Моделирование и анализ информационных систем. №~2~(28). 2021. С.~198--214. DOI:~\href{https://doi.org/10.18255/1818-1015-2021-2-198-214}{10.18255/1818-1015-2021-2-198-214}

\bibitem{backus}
Backus, J.
Can programming be liberated from von Neuman style? A functional
stile and its algebra of programs. CACM 21(8), 613–641 (1978). DOI:~\href{https://doi.org/10.1145/359576.359579}{10.1145/359576.359579}

\bibitem{smile}
Alexander Legalov, Igor Legalov, Ivan Matkovskii. Specifics of Semantics of a Statically Typed Language of Functional and Dataflow Parallel Programming - Scientific Services \& Internet 2019.  CEUR Workshop Proceedings, Vol. 2543. P. 274--284. DOI:~\href{https://doi.org/10.20948/abrau-2019-08}{10.20948/abrau-2019-08}.

\bibitem{dyn}
Legalov~A.\,I., Matkovskii~I.\,V., Ushakova~M.\,S., Romanova~D.\,S. Dynamically Changing Parallelism with Asynchronous Sequential Data Flows // Automatic Control and Computer Sciences. 2021. Vol.~55, No.~7. P.~636--646. DOI:~\href{https://doi.org/10.3103/S0146411621070105}{10.3103/S0146411621070105}

\bibitem{stat-model}
Легалов А.И., Чуйкин Н.К. Поддержка статической типизации в функционально–потоковой модели параллельных вычислений // Параллельные вычислительные технологии – XVII всероссийская конференция с международным участием, ПаВТ'2023, г. Санкт-Петербург, 28–30 марта 2023 г. DOI:~\href{https://doi.org/10.14529/pct2023}{10.14529/pct2023}. C. 173–185. Электронная версия статьи размещена по адресу: \href{http://omega.sp.susu.ru/pavt2023/short/024.pdf}{http://omega.sp.susu.ru/pavt2023/short/024.pdf}
\end{thebibliography}
